%% packages and libraries
\usepackage{amsmath, amssymb, amsfonts, mathtools, cases, mathrsfs}	%math
\usepackage{physics}
\usepackage{fancyhdr, lastpage}										%header and footer
%\usepackage{titlesec}													%title design
\usepackage{hyperref}													%link
\usepackage{url}														%URL
\usepackage{pxjahyper}												%arrow Japanese for hyperref
\usepackage[table]{xcolor}											%tables and color
\usepackage{longtable, float, multirow, array, listliketab, enumitem, tabularx}	%tables
\usepackage{flafter}
\usepackage{comment}												%comment out over 2 lines
\usepackage{graphicx}													%figure
\usepackage{subcaption, wrapfig}									%configuration
\usepackage{tikz}														%TikZ
\usetikzlibrary{calc, patterns, decorations, decorations.pathmorphing, angles, backgrounds, shadows, positioning, shapes.geometric, arrows}
\usepackage{algorithm, algorithmic}
\usepackage{pdfpages}												%include PDF
\usepackage{import, grffile}											%management files
\usepackage{standalone}
\usepackage[T1]{fontenc}											%fonts
\usepackage{textcomp}
\usepackage[utf8]{inputenc}
\usepackage{lmodern}
\usepackage{mathptmx}
\usepackage[scaled]{helvet}
\renewcommand{\ttdefault}{pcr}
\usepackage[deluxe]{otf}
\usepackage[noalphabet]{pxchfon}
\usepackage{bm}
\usepackage{lscape}
\usepackage{siunitx}													%units
\usepackage{bigstrut}


%% set up for hyperref
\hypersetup{
	bookmarksnumbered = true,
	hidelinks,
	colorlinks = true,
	linkcolor = black,
	urlcolor = cyan,
	citecolor = black,
	filecolor = magenta,
	setpagesize = false,
	pdftitle = {},
	pdfauthor = {Morita},
	pdfkeywords = {},
	}


%% set up for siunitx
\sisetup{%
	%detect-family = true,						%bold
	detect-inline-family = math,
	detect-weight = true,
	detect-inline-weight = math,
	%input-product = *,							%multiply by *
	quotient-mode = fraction,						%fraction by /
	fraction-function = \frac,
	inter-unit-product = \ensuremath{\hspace{-1.5pt}\cdot\hspace{-1.5pt}},	%devide different units by \cdot
	per-mode = symbol,%							%devide different unit by /
	product-units = single,%						%
	}


%% fonts
\setminchofont{ipam.ttf}
\setgothicfont{ipag.ttf}
\fontsize{14Q}{30H}


%% layout
%space between letters
\kanjiskip 0zw plus 1zw minus 0zw
\xkanjiskip 0.25zw plus 1pt minus 1pt
%line skip
\setlength{\lineskiplimit}{2pt}
\setlength{\lineskip}{2pt}
%indent
\setlength{\parindent}{1zw}
\setlength{\mathindent}{5zw}


%put citation mark on index zone
\bibliographystyle{junsrt}


%hyphen for math mode
\DeclareMathSymbol{\mhyphen}{\mathalpha}{operators}{`-}


%circle text
\newcommand{\ctext}[1]{\raise0.2ex\hbox{\textcircled{\scriptsize{#1}}}}


%unit of formulas 〔roman〕
\newcommand{\unitis}[1]{%
	\text{〔\si{#1}〕}
	}

