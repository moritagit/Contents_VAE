\documentclass[dvipdfmx, fleqn]{jsarticle}
%% packages and libraries
\usepackage{amsmath, amssymb, amsfonts, mathtools, cases, mathrsfs}	%math
\usepackage{physics}
\usepackage{fancyhdr, lastpage}										%header and footer
%\usepackage{titlesec}													%title design
\usepackage{hyperref}													%link
\usepackage{url}														%URL
\usepackage{pxjahyper}												%arrow Japanese for hyperref
\usepackage[table]{xcolor}											%tables and color
\usepackage{longtable, float, multirow, array, listliketab, enumitem, tabularx}	%tables
\usepackage{flafter}
\usepackage{comment}												%comment out over 2 lines
\usepackage{graphicx}													%figure
\usepackage{subcaption, wrapfig}									%configuration
\usepackage{tikz}														%TikZ
\usetikzlibrary{calc, patterns, decorations, decorations.pathmorphing, angles, backgrounds, shadows, positioning, shapes.geometric, arrows}
\usepackage{algorithm, algorithmic}
\usepackage{pdfpages}												%include PDF
\usepackage{import, grffile}											%management files
\usepackage{standalone}
\usepackage[T1]{fontenc}											%fonts
\usepackage{textcomp}
\usepackage[utf8]{inputenc}
\usepackage{lmodern}
\usepackage{mathptmx}
\usepackage[scaled]{helvet}
\renewcommand{\ttdefault}{pcr}
\usepackage[deluxe]{otf}
\usepackage[noalphabet]{pxchfon}
\usepackage{bm}
\usepackage{lscape}
\usepackage{siunitx}													%units
\usepackage{bigstrut}


%% set up for hyperref
\hypersetup{
	bookmarksnumbered = true,
	hidelinks,
	colorlinks = true,
	linkcolor = black,
	urlcolor = cyan,
	citecolor = black,
	filecolor = magenta,
	setpagesize = false,
	pdftitle = {},
	pdfauthor = {Morita},
	pdfkeywords = {},
	}


%% set up for siunitx
\sisetup{%
	%detect-family = true,						%bold
	detect-inline-family = math,
	detect-weight = true,
	detect-inline-weight = math,
	%input-product = *,							%multiply by *
	quotient-mode = fraction,						%fraction by /
	fraction-function = \frac,
	inter-unit-product = \ensuremath{\hspace{-1.5pt}\cdot\hspace{-1.5pt}},	%devide different units by \cdot
	per-mode = symbol,%							%devide different unit by /
	product-units = single,%						%
	}


%% fonts
\setminchofont{ipam.ttf}
\setgothicfont{ipag.ttf}
\fontsize{14Q}{30H}


%% layout
%space between letters
\kanjiskip 0zw plus 1zw minus 0zw
\xkanjiskip 0.25zw plus 1pt minus 1pt
%line skip
\setlength{\lineskiplimit}{2pt}
\setlength{\lineskip}{2pt}
%indent
\setlength{\parindent}{1zw}
\setlength{\mathindent}{5zw}


%put citation mark on index zone
\bibliographystyle{junsrt}


%hyphen for math mode
\DeclareMathSymbol{\mhyphen}{\mathalpha}{operators}{`-}


%circle text
\newcommand{\ctext}[1]{\raise0.2ex\hbox{\textcircled{\scriptsize{#1}}}}


%unit of formulas 〔roman〕
\newcommand{\unitis}[1]{%
	\text{〔\si{#1}〕}
	}


\begin{document}


\begin{longtable}{p{4cm}p{4cm}p{7cm}}
    %\centering
    \caption{用語集}
    \label{tab:terms}
    \\ \hline
    単語 & 日本語訳 & 意味 \tabularnewline \hline \hline
    \endfirsthead
    単語 & 日本語訳 & 意味 \tabularnewline \hline \hline
    \endhead
        autoencoder
            & 自己符号化器
            & ニューラルネットワークにより,入力を圧縮したのちそれを復元するように学習したモデルのこと。
            \tabularnewline \hline
        auto-encoding
            & 自己符号化
            & 
            \tabularnewline \hline
        variational
            & 変分
            & 
            \tabularnewline \hline
        Variational Bayes
            & 変分ベイズ
            & 
            \tabularnewline \hline
        inference
            & 推論
            & モデルにデータを入れて出力を得ること
            \tabularnewline \hline
        directed probabilistic models
            & 有向確率モデル
            & 有向グラフで定義される確率モデル
            \tabularnewline \hline
        continuous
            & 連続な
            & 離散的な(discrete)の対義語
            \tabularnewline \hline
        latent variable
            & 潜在変数
            & 人からは見えない,実際に観測されない,という意味で「潜在」的な変数。
            観測されるデータを「結果」として考えたときの,
            「原因」に当たるものとして用いられることが多い(生成モデル)。
            \tabularnewline \hline
        dataset
            & ひとまとまりのデータ
            & 
            \tabularnewline \hline
        intractable
            & イントラクタブル・計算不可能
            & 理論上計算可能な問題であっても,
            計算量が大きすぎて実質的に計算できない。
            具体的には指数時間のアルゴリズムでかつ\(n\)が大きいとき。
            \tabularnewline \hline
        efficient
            & 計算可能な
            & 直訳では「効率的な」という意味であるが,
            ここではintractableの反対語のような役割で,
            計算量的な観点で計算可能な,といった意味合い。
            \tabularnewline \hline
        posterior distribution
            & 事後分布
            & 事後確率分布(posterior probability distribution)の略。
            posteriorと略されることもある。
            \tabularnewline \hline
        stochastic
            & 確率的な
            & 
            \tabularnewline \hline
        variational lower bound
            & 変分下限
            & 
            \tabularnewline \hline
        stochastic gradient method
            & 確率的勾配法
            & 確率的勾配上昇法と確率的勾配降下法をまとめて表現した語句。
            \tabularnewline \hline
        i.i.d.
            & 独立同分布
            & 「independent and identically distributed」の略。
            それぞれの確率変数が他の確率変数と同じ確率分布を持ち,
            かつそれぞれ互いに独立していること。
            \tabularnewline \hline
        datapoint
            & データ点・データ要素
            & データセット中のある1つのデータ
            \tabularnewline \hline
        mean-field
            & 平均場
            & 
            \tabularnewline \hline
        w.r.t
            & ~に関して
            & 「with respect to」の略
            \tabularnewline \hline
        ancestral sampling
            & 伝承サンプリング
            & 
            \tabularnewline \hline
        expensive
            & 計算コストの高い
            & 
            \tabularnewline \hline
        MCMC
            & マルコフ連鎖モンテカルロ法
            & 
            \tabularnewline \hline
        maximum likelihood inference
            & 最尤推定
            & 頭文字を取ってMLと略される
            (maximum likelihood estimationをMLEと略す場合が多い)
            \tabularnewline \hline
        maximum a posteriori inferecne
            & 最大事後確率推定
            & 頭文字を取ってMAP推定と略される
            \tabularnewline \hline
        discrete
            & 離散的な
            & continuousの対義語
            \tabularnewline \hline
        random process
            & 確率過程
            & 
            \tabularnewline \hline
        random variable
            & 確率変数
            & 
            \tabularnewline \hline
        conditional distribution
            & 条件付き確率
            & 
            \tabularnewline \hline
        prior
            & 事前分布
            & 
            \tabularnewline \hline
        likelihood
            & 尤度
            & 
            \tabularnewline \hline
        parametric family
            & 
            & 
            \tabularnewline \hline
        PDF
            & 確率密度関数
            & 「probability density function」の略
            \tabularnewline \hline
        marginal probability
            & 周辺確率
            & 
            \tabularnewline \hline
        marginal likelihood
            & 周辺尤度
            & 
            \tabularnewline \hline
        EM algorythm
            & EMアルゴリズム
            & 
            \tabularnewline \hline
        coding
            & 符号化
            & 
            \tabularnewline \hline
        data representation task
            & 表現学習
            & feature learningと同義
            \tabularnewline \hline
        denoising
            & ノイズ除去
            & 
            \tabularnewline \hline
        inpainting
            & 修復
            & 画像の一部が欠けているとき,
            周りのピクセルの値からその欠損を埋めるようなタスク
            \tabularnewline \hline
        super-resolution
            & 超解像
            & 画質の低い画像を高画質にするタスク
            \tabularnewline \hline
        factorial
            & 因子分解可能な
            & 
            \tabularnewline \hline
        closed-form
            & 閉形式
            & 
            \tabularnewline \hline
        expectation
            & 期待値
            & 
            \tabularnewline \hline
        coding theory
            & 符号理論
            & 
            \tabularnewline \hline
        encoder
            & 符号器
            & 
            \tabularnewline \hline
        decoder
            & 復号器
            & 
            \tabularnewline \hline
        RHS
            & (数式の)右辺
            & right-hand side
            \tabularnewline \hline
        KL divergence
            & カルバック・ライブラー情報量
            & 2つの確率分布の差異を測る尺度。
            距離の定義には当てはまらない。
            KLはKullback–Leiblerの略。
            \tabularnewline \hline
        minibatch
            & ミニバッチ
            & 
            \tabularnewline \hline
        objective function
            & 目的関数
            & 学習の際に最適化する関数
            \tabularnewline \hline
        regularizer
            & 正則化項
            & 目的関数のうち,汎化性能の向上を意図して用いられている項。
            なお,パラメータの値に制限をかけることでこれを行っている。
            \tabularnewline \hline
        tractable
            & (多項式時間で)計算可能な
            & intractableの対義語
            \tabularnewline \hline
        inverse CDF
            & 逆累積分布関数
            & ICDFとも。
            CDFはcumulative distribution functionの略。
            \tabularnewline \hline
        centered isotropic multivariate Gaussian
            & 標準多変量ガウス分布
            & centeredは平均が\(\bm{0}\)であることを,isotropic(等方性)は共分散行列が対角成分しか持たず,その値が全て等しいことを表す。
            \tabularnewline \hline
        MLP
            & 多層パーセプトロン
            & multi layer perceptronの略。
            \tabularnewline \hline
        computational complexity
            & 計算量・計算複雑性
            & 
            \tabularnewline \hline
        exponential family
            & 
            & 
            \tabularnewline \hline
        natural parameter
            & 
            & 
            \tabularnewline \hline
        infomax principle
            & 情報量最大化原理
            & 
            \tabularnewline \hline
        mutual information
            & 相互情報量
            & 
            \tabularnewline \hline
        conditional entropy
            & 条件付きエントロピー
            & 
            \tabularnewline \hline
        contractive
            & 収縮的な
            & 
            \tabularnewline \hline
        sparse
            & 疎な
            & 
            \tabularnewline \hline
        hyperparameter
            & ハイパーパラメータ
            & 人手で調整されるパラメータ。
            最適化の対象となる,モデルの重みなどは単にパラメータと呼ばれ,
            これらは区別される。
            \tabularnewline \hline
        weight decay
            & 重み減衰
            & パラメータのL2正則化をする項
            \tabularnewline \hline
        stepsize
            & 学習率
            & learning rateと呼ばれることも多い
            \tabularnewline \hline
        annealing
            & 焼きなまし法
            & 学習率を段々小さくしていく方法
            \tabularnewline \hline
\end{longtable}


\end{document}
